% SiCP
%
% Librement inspiré de
% TeXtronics oscilloscope
% Author: Hugues Vermeiren
% http://www.texample.net/tikz/examples/textronics-oscilloscope/
\documentclass[a4paper,12pt,oneside]{article}
\usepackage{tikz}
%%%<
\usepackage{verbatim}
\usepackage[active,tightpage]{preview}
\usepackage[T1]{fontenc}
\usepackage[utf8]{inputenc}
%\usepackage[greek, frenchb]{babel}
\PreviewEnvironment{tikzpicture}
\setlength\PreviewBorder{0pt}%
%%%>
\begin{comment}
:Title: TeXtronics oscilloscope
:Tags: Styles;Scopes;Physics; Fun
:Author: Hugues Vermeiren
:Slug: textronics-oscilloscope
The TeXtronics oscilloscope! It is a basic 2-channels oscilloscope,
the kind of machine most students have used in their labs.

Ideally, it should be parameterized so as to be able to easily change
the signals and the positions of the different controls.
\end{comment}

\begin{document}
\def\scl{0.6}%scaling factor of the picture
\sffamily
\begin{tikzpicture}[
  scale=\scl,
  panneaucontroles/.style={yellow!30!brown!20!,rounded corners,draw=black,thick},
  lumiereeteinte/.style={yellow!40!brown!30!,draw=yellow!40!brown!60!,thin},
  petitbouton/.style={white,draw=black, thick}]
%
%
%        ECRAN
%
    \fill[white,draw=black,thick](-21,-0.7) rectangle (28,30.9);
%
	% CAPTEURS HAUT
    \def\hauteur{7.7}
%   SPECTRE GAUCHE
  \begin{scope}[xshift=-16.7 cm,yshift=\hauteur cm]
    % Draw axes
    \draw [<->,draw=gray,thick] (0,7) node (yaxis) [above] {$gauche$}
        |- (16,0) node (xaxis) [right] {$f$};
  \end{scope}
%   SPECTRE DROITE
  \begin{scope}[xshift=1.3 cm,yshift=\hauteur cm]
    % Draw axes
    \draw [<->,draw=gray,thick] (0,7) node (yaxis) [above] {$droite$}
        |- (16,0) node (xaxis) [right] {$f$};
  \end{scope}
%
%      PANNEAU  BAS
%
  \fill[panneaucontroles] (-20.5,-0.3) rectangle (19.5,3.7);
%
%       Vitesse de la simulation
  \begin{scope}[xshift=-11.5 cm,yshift=-0.3cm]
   % \fill[red,draw=yellow!40!red!60!,thick] (0.0,0.8) circle (0.3cm);
    \node[scale=\scl] at (6.5,3.3) {\LARGE \textbf{Simulation}};

    \draw[draw=black!20!brown!80!,very thick] (2.0,0.8)-- (4.1,0.8);

    \fill[blue!50!black!20!,draw=black,thick,rounded corners=6pt] (2.5,1.45)-- (2.5,2.55)-- (1.6,2.0)-- cycle;
    \fill[blue!50!black!20!,draw=black,thick,rounded corners=6pt] (2.8,1.45)-- (2.8,2.55)-- (1.9,2.0)-- cycle;

    \fill[blue!50!black!20!,draw=black,thick,rounded corners=6pt] (4.4,1.45)-- (4.4,2.55)-- (3.5,2.0)-- cycle;

    \fill[blue!50!black!20!,draw=black,thick,rounded corners=2pt] (5.35,1.6) rectangle (5.7,2.4);
    \fill[blue!50!black!20!,draw=black,rounded corners=2pt] (6.5,2.0)-- (5.8,1.55)-- (5.8,2.45)-- cycle;
    \fill[lumiereeteinte] (5.75,0.8) circle (0.3cm);

    \fill[blue!50!black!20!,draw=black,rounded corners=1pt] (8.1,2.0)-- (7.3,1.5)-- (7.3,2.5)-- cycle;
    \fill[lumiereeteinte] (7.7,0.8) circle (0.3cm);

    \fill[blue!50!black!20!,draw=black,thick,rounded corners=6pt] (9.9,2.0)-- (9.0,1.45)-- (9.0,2.55)-- cycle;

    \fill[blue!50!black!20!,draw=black,thick,rounded corners=6pt] (11.9,2.0)-- (11.0,1.45)-- (11.0,2.55)-- cycle;
    \fill[blue!50!black!20!,draw=black,thick,rounded corners=6pt] (11.7,2.0)-- (10.8,1.45)-- (10.8,2.55)-- cycle;

    \draw[draw=black!20!brown!80!,very thick] (9.3,0.8)-- (11.5,0.8);

  \end{scope}
%
%
%       Réinitialisaation
  \begin{scope}[xshift=3.5 cm,yshift=-0.3cm]

  %panneaucontroles/.style={yellow!30!brown!20!,rounded corners,,thick},
  \draw[blue,rounded corners,thin] (-0.6,0.2) rectangle (15.8,3.8);

    \node[scale=\scl] at (9.1,0.8) {\LARGE \textbf{Initialisation}};

    \node[scale=\scl] at (3.7,3.3) {\Large \textbf{Position}};

    \fill[blue!50!black!20!,draw=black,thick] (0.5,2.0) circle (0.43cm);
    \node[scale=\scl] at (0.5,2.0) {\Large \textbf{1}};
    \fill[blue!50!black!20!,draw=black,thick] (1.9,2.0) circle (0.43cm);
    \node[scale=\scl] at (1.9,2.0) {\Large \textbf{2}};
    \fill[blue!50!black!20!,draw=black,thick] (3.3,2.0) circle (0.43cm);
    \node[scale=\scl] at (3.3,2.0) {\Large \textbf{3}};
    \fill[blue!50!black!20!,draw=black,thick] (4.7,2.0) circle (0.43cm);
    \node[scale=\scl] at (4.7,2.0) {\Large \textbf{4}};
    \fill[blue!50!black!20!,draw=black,thick] (6.1,2.0) circle (0.43cm);
    \node[scale=\scl] at (6.1,2.0) {\Large \textbf{5}};
    \fill[blue!50!black!20!,draw=black,thick] (7.5,2.0) circle (0.43cm);
    \node[scale=\scl] at (7.5,2.0) {\Large \textbf{6}};

    \node[scale=\scl] at (12.5,3.3) {\Large \textbf{Parametres}};

    \fill[blue!50!black!20!,draw=black,thick] (10.4,2.0) circle (0.43cm);
    \node[scale=\scl] at (10.4,2.0) {\Large \textbf{1}};
    \fill[blue!50!black!20!,draw=black,thick] (11.8,2.0) circle (0.43cm);
    \node[scale=\scl] at (11.8,2.0) {\Large \textbf{2}};
    \fill[blue!50!black!20!,draw=black,thick] (13.2,2.0) circle (0.43cm);
    \node[scale=\scl] at (13.2,2.0) {\Large \textbf{3}};
    \fill[blue!50!black!20!,draw=black,thick] (14.6,2.0) circle (0.43cm);
    \node[scale=\scl] at (14.6,2.0) {\Large \textbf{4}};

  \end{scope}
%
%      PANNEAU  COUPLAGE
%
  \fill[panneaucontroles] (20.1,24.2) rectangle (27.6,30.4);
  \begin{scope}[xshift=22.7 cm,yshift=26.9cm]
    \fill[petitbouton] (3.5,-2.05) circle (0.3cm);
    \node[scale=\scl] at (3.5,-1.35) {\Large{Mixte}};
    \fill[petitbouton] (3.5,-0.55) circle (0.3cm);
    \node[scale=\scl] at (3.5,0.05) {\Large{Fixe}};
    \fill[petitbouton] (3.5,0.95) circle (0.3cm);
    \node[scale=\scl] at (3.5,1.55) {\Large{Libre}};
    \fill[petitbouton] (3.5,2.35) circle (0.3cm);
    \node[scale=\scl] at (3.5,2.95) {\Large{Périodique}};

    \node[scale=\scl] at (0,2.6) {\LARGE \textbf{Couplage}};
    \clip[rounded corners] (-2,-2) rectangle (2,2);
    \fill[black!30!,rounded corners,draw=black,thick] (-2,-2) rectangle (2,2);
    \draw[very thick,rounded corners](-2,-2) rectangle (2,2);
    \draw[thick] (0,0) circle (1.0);
    \foreach \i in {0,30,...,330} \draw[thick] (\i:1.2)--(\i:2.5);
    \foreach \i/\j in {15/1.5,45/1,75/.5,105/.3,135/.2,165/.1,195/.05,225/.03,255/.02,285/10,315/6,345/3} \node[scale=\scl,black] at (\i:1.7) {\j};
    \fill[blue!30!black!60!,draw=black,thick] (0,0) circle (0.8cm);
  \end{scope}
%
%      PANNEAU  DISSIPATION
%
  \fill[panneaucontroles] (20.1,18) rectangle (27.6,23.7);
  \begin{scope}[xshift=22.7 cm,yshift=20.5cm]
    \fill[petitbouton] (3.5,-1.5) circle (0.3cm);
    \node[scale=\scl] at (3.5,-0.85) {\Large{Extrémité}};
    \fill[petitbouton] (3.5,0.1) circle (0.3cm);
    \node[scale=\scl] at (3.5,0.8) {\Large{Nulle}};
    \fill[petitbouton] (3.5,1.7) circle (0.3cm);
    \node[scale=\scl] at (3.5,2.4) {\Large{Uniforme}};

    \node[scale=\scl] at (0,2.6) {\LARGE \textbf{Dissipation}};
    \clip[rounded corners] (-2,-2) rectangle (2,2);
    \fill[black!30!,rounded corners,draw=black,thick] (-2,-2) rectangle (2,2);
    \draw[very thick,rounded corners](-2,-2) rectangle (2,2);
    \draw[thick] (0,0) circle (1.0);
    \foreach \i in {0,30,...,330} \draw[thick] (\i:1.2)--(\i:2.5);
    \foreach \i/\j in {15/4,45/2.3,75/1.3,105/.7,135/.4,165/.25,195/.1,225/.07,255/.04,285/25,315/13,345/7} \node[scale=\scl,black] at (\i:1.7) {\j};
    \fill[blue!30!black!60!,draw=black,thick] (0,0) circle (0.8cm);
  \end{scope}
%
%      PANNEAU  MOTEUR
%
  \fill[panneaucontroles] (20.2,6.5) rectangle (27.5,17.5);
 % \fill[panneaucontroles] (20,13) rectangle (27,19.5);
  \begin{scope}[xshift=26.2 cm,yshift=6.7cm]
    \fill[petitbouton] (0,2.8) circle (0.3cm);
    \node[scale=\scl] at (0,3.5) {\Large{Impulsion}};
    \fill[petitbouton] (0,4.4) circle (0.3cm);
    \node[scale=\scl] at (0,5.1) {\Large{Carré}};
    \fill[petitbouton] (0,6.0) circle (0.3cm);
    \node[scale=\scl] at (0,6.7) {\Large{Sinus}};
    \fill[petitbouton] (0,7.7) circle (0.3cm);
    \node[scale=\scl] at (0,8.4) {\Large{Arrêt}};
  \end{scope}
  \begin{scope}[xshift=22.7 cm,yshift=14.3cm]%,scale=0.85
    \node[scale=\scl] at (0,2.6) {\LARGE \textbf{Amplitude}};
    \clip[rounded corners] (-2,-2) rectangle (2,2);
    \fill[black!30!,rounded corners,draw=black,thick] (-2,-2) rectangle (2,2);
    \draw[very thick,rounded corners](-2,-2) rectangle (2,2);
    \draw[thick] (0,0) circle (1.0);
    \foreach \i in {0,30,...,330} \draw[thick] (\i:1.2)--(\i:2.5);
    \foreach \i/\j in {15/1.7,45/1.2,75/.9,105/.6,135/.4,165/.3,195/.2,225/.1,255/.07,285/7,315/4,345/2.7} \node[scale=\scl,black] at (\i:1.7) {\j};
    \fill[blue!30!black!60!,draw=black,thick] (0,0) circle (0.8cm);
  \end{scope}
  \begin{scope}[xshift=22.7 cm,yshift=9.1cm,scale=0.90]
    \node[scale=\scl] at (0,2.6) {\LARGE \textbf{Fréquence}};
    \clip[rounded corners] (-2,-2) rectangle (2,2);
    \fill[black!30!,rounded corners,draw=black,thick] (-2,-2) rectangle (2,2);
    \draw[very thick,rounded corners](-2,-2) rectangle (2,2);
    \draw[thick] (0,0) circle (1.0);
    \foreach \i in {0,30,...,330} \draw[thick] (\i:1.2)--(\i:2.5);
    \foreach \i/\j in {15/10,45/5,75/3,105/1.7,135/1,165/.6,195/.3,225/.2,255/.1,285/60,315/30,345/17} \node[scale=\scl,black] at (\i:1.7) {\j};
    \fill[blue!30!black!60!,draw=black,thick] (0,0) circle (0.8cm);
  \end{scope}
%
%      PANNEAU  JOSEPHSON
%
  \fill[panneaucontroles] (20.1,-0.2) rectangle (27.6,6.0);
  \begin{scope}[xshift=22.7 cm,yshift=2.1cm]
    \fill[petitbouton] (3.5,-0.55) circle (0.3cm);
    \node[scale=\scl] at (3.5,0.05) {\Large{Sens}};
    \fill[petitbouton] (3.5,0.95) circle (0.3cm);
    \node[scale=\scl] at (3.5,1.55) {\Large{Arrêt}};
    \fill[petitbouton] (3.5,2.35) circle (0.3cm);
    \node[scale=\scl] at (3.5,2.95) {\Large{Marche}};
  \end{scope}

  \begin{scope}[xshift=22.7 cm,yshift=2.6cm,scale=0.85]
    \node[scale=\scl] at (0,2.6) {\LARGE \textbf{Josephson}};
    \clip[rounded corners] (-2,-2) rectangle (2,2);
    \fill[black!30!,rounded corners,draw=black,thick] (-2,-2) rectangle (2,2);
    \draw[very thick,rounded corners](-2,-2) rectangle (2,2);
    \draw[thick] (0,0) circle (1.0);
    \foreach \i in {0,30,...,330} \draw[thick] (\i:1.2)--(\i:2.5);
    \foreach \i/\j in {15/9,45/5,75/3,105/1.8,135/1,165/.6,195/.3,225/.2,255/.1,285/60,315/30,345/18} \node[scale=\scl,black] at (\i:1.7) {\j};
    \fill[blue!30!black!60!,draw=black,thick] (0,0) circle (0.8cm);

  \end{scope}

%
\end{tikzpicture}
%
%%%%%%%%%%%%%%%%%%%%%%%%%%%%%%%%%%%%%%%%%%%%%%%%%%%%%%%%%%%%%%%%%%%%%%%%%%%%%%%%%%
\end{document}
%%%%%%%%%%%%%%%%%%%%%%%%%%%%%%%%%%%%%%%%%%%%%%%%%%%%%%%%%%%%%%%%%%%%%%%%%%%%%%%%%%
